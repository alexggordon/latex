\documentclass[12]{scrartcl}
\usepackage{amssymb,amsmath,gensymb,dsfont,calc,multicol,fullpage}
\makeatletter
\newcommand\Aboxed[1]{
   \@Aboxed#1\ENDDNE}
\def\@Aboxed#1&#2\ENDDNE{%
   &
   \settowidth\@tempdima{$\displaystyle#1{}$}
   \setlength\@tempdima{\@tempdima+\fboxsep+\fboxrule}
   \kern-\@tempdima
   \boxed{#1#2}
}
\makeatother

\begin{document}

\title{Homework 27, Section 5.1: 18, 21, 23, 25, 31}
\author{Alex Gordon}
\date{\today}
\maketitle
\section*{Homework}
\subsection*{18.}
Upper triangular matrix, so the eigenvalues are the values on the diagonal $= 5, 0, 3$
\subsection*{21. A)}
False. The equation $Ax = \lambda x$ must have a nontrivial solution. 
\subsection*{21. B)}
True. If 0 is an eigenvalue then is equivalent to $Ax = 0$, which has a nontrivial solution if and only if A is not invertible. Thus 0 is an eigenvalue of A if and only if A is not invertible. 
\subsection*{21. C)}
True. The set of all solutions of $A - \lambda I)x = 0$ is just the null space of the matrix $Ax = \lambda I$. 
\subsection*{21. D)}
True. To check if it is an eigenvector, just multiply it by the matrix in question. 
\subsection*{21. E)}
True. Although row reduction is used to find eigenvectors, it cannot be used to find eigenvalues. 
\subsection*{23.}
If a 2 x 2 matrix A were to have three distinct eigen values then by theorem 2 there would correspond three linearly independent eigenvectors. This is impossible because the vectors all belong to a two-dimensional vector space, in which any set of three vectors in linearly dependent. 
\subsection*{25.}
If $\lambda$ is an eigenvalue of A then there is a nonzero vector x such that $Ax - \lambda x$. Since $x \neq 0$, $\lambda$ cannot be zero. This then means that $\lambda^{-1} Ax = A^{-1}x$, which shows that $\lambda^{-1}$ is an eigenvalue of $A^{-1}$. 
\subsection*{31.}
Suppose T reflects points through a line that passes through the origin of the matrix. The line is multiples of some nonzero vector v, so the points on the line do not move under A. So T(v) = v. If A is the standard matrix of T, then $Av = v$. Thus v is an eigenvector of A corresponding to the eigenvalue 1. This also means another eigenspace is generated by any nonzero vector u that is perpendicular to the given line. This, by the properties of transformation, each vector on the line through the vector u is transformed onto the vector -x, meaning the eigenvalue is -1. 
\end{document}