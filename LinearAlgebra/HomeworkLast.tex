\documentclass[12]{scrartcl}
\usepackage{amssymb,amsmath,gensymb,dsfont,calc,multicol,fullpage}
\makeatletter
\newcommand\Aboxed[1]{
   \@Aboxed#1\ENDDNE}
\def\@Aboxed#1&#2\ENDDNE{%
   &
   \settowidth\@tempdima{$\displaystyle#1{}$}
   \setlength\@tempdima{\@tempdima+\fboxsep+\fboxrule}
   \kern-\@tempdima
   \boxed{#1#2}
}
\makeatother

\begin{document}

\title{The Last Homework Item I'll Turn In During College, Section 6.5: 17, 18}
\author{Alex Gordon}
\date{\today}
\maketitle
\section*{Freedom}
\subsection*{17. A)}
True. If $m x n$  and B is in $R^M$, a least squares solution of $Ax = b$ is an $\hat{x} $ in $R^n$. 
\subsection*{17. B)}
True. The projection gives us the best approximation
\subsection*{17. C)}
False. The inequality is facing the wrong way. 
\subsection*{17. D)}
True. This is how we can find the least squares solutions
\subsection*{17. E)}
True. Then $A^TA $ is invertible so we can solve $A^TAx = A^Tb $ for x by taking the inverse.  

\subsection*{18. A)}
True. See the paragraph after the least squares solution. 
\subsection*{18. B)}
False. If $\hat{x}$ is the least squares solution then $A \hat{x}$ is the point in the column space closest to b. 
\subsection*{18. C)}
True. See the discussion following equation 1
\subsection*{18. D)}
False. The formula applies only when the columns of A are linearly independent.  
\subsection*{18. E)}
False. See the comments after example 4.
\subsection*{18. F)}
False. Eq. (7) numberical note. 



\end{document}