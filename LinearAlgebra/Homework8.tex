\documentclass[12]{scrartcl}
\usepackage{amssymb,amsmath,gensymb,dsfont,calc,multicol,fullpage}
\makeatletter
\newcommand\Aboxed[1]{
   \@Aboxed#1\ENDDNE}
\def\@Aboxed#1&#2\ENDDNE{%
   &
   \settowidth\@tempdima{$\displaystyle#1{}$}
   \setlength\@tempdima{\@tempdima+\fboxsep+\fboxrule}
   \kern-\@tempdima
   \boxed{#1#2}
}
\makeatother

\begin{document}

\title{Homework 8, Section 1.8; 4, 7, 8, 9, 15, 26, 30}
\author{Alex Gordon}
\date{\today}
\maketitle
\section*{Homework}
\subsection*{4.}
RREF Matrix results in:
$\begin{bmatrix} 1&0&0&-17 \\0&1&0&-7 \\ 0&0&1&-1  \end{bmatrix}$\\
So $x_1 = -17$\\
$x_2 = -7$\\
$x_3 = -1$\\
and $ X = \begin{bmatrix} -17 \\-7 \\ -1  \end{bmatrix}$\\
This makes it clear that X is unique.
\subsection*{7.}
Let A be a 6x5 matrix. What must a and b be in order to define $T : \mathbb{R}^a \rightarrow \mathbb{R}^b$ by $T(X) = Ax$.
The matrix A and the column vector x are multipliable. If A is a matrix of order p x q, and x is a matrix of order q x l, then the product matrix is of order p x l. 
This confirms that the dimension of range space is p, meaning the number of rows of A is the dimension of the range vector space. 
Since 6 is the dimension of the range space and 5 is the dimension of the domain vector space, then A is a matrix of order 6 x 5 and $T : \mathbb{R}^a \rightarrow \mathbb{R}^b$ by $T(X) = Ax$ that is A has $b = 6$ rows and $a = 5$ columns and so the co domain of $T$ is $\mathbb{R}^6$ and the domain of $T$ is $\mathbb{R}^5$.
Therefore, $a =5$ and $b = 6$.
\subsection*{8.}
Since $X \in \mathbb{R}^5$, we have x has order $ 5 \cdot 1$ \\
Since $Ax \in \mathbb{R}^7$, we have $Ax$ with order $ 7 \cdot 1$.
Thus, by the product rule of matrices the number of rows in $Ax$ and the number of columns in $X$ equal the number of rows and columns in A, so A must be a matrix with 7 rows and 5 columns. 
\subsection*{9.}
RREF Matrix results in:
$\begin{bmatrix} 1&0&-4&10&0 \\0&1&-3&5&0 \\ 0&0&0&4&0  \end{bmatrix}$\\
Hence $x_4 = 0$\\
$x_3 = free$\\
$x_2 = 3x_3$\\
$x_1 = 4x_3$\\
and $ X = \begin{bmatrix} 4 \\3 \\ 1 \\ 0 \end{bmatrix}$\\
\subsection*{15.}
$ T(u) = \begin{bmatrix} 2  \\ 5  \end{bmatrix}$\\
$ T(5) = \begin{bmatrix} 4  \\ -2  \end{bmatrix}$\\
The images of $T(u)$ and $ T(v)$ are a reflection through the line $x_2 = x_1$
\subsection*{26.}

\subsection*{30.}
$T(x) = c_1, \cdot 0 + c_2 \cdot 0 + ...... + c_p \cdot 0$\\
Therefore, $T(x) = 0$ and T is linear. 



\end{document}