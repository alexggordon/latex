\documentclass[12]{scrartcl}
\usepackage{amssymb,amsmath,gensymb,dsfont,calc,multicol,fullpage}
\makeatletter
\newcommand\Aboxed[1]{
   \@Aboxed#1\ENDDNE}
\def\@Aboxed#1&#2\ENDDNE{%
   &
   \settowidth\@tempdima{$\displaystyle#1{}$}
   \setlength\@tempdima{\@tempdima+\fboxsep+\fboxrule}
   \kern-\@tempdima
   \boxed{#1#2}
}
\makeatother

\begin{document}

\title{Homework 4, Section 1.4: 4, 5, 14, 17, 19, 21, 22, 24, 31, 35}
\author{Alex Gordon}
\date{\today}
\maketitle
\section*{Homework}
\subsection*{4. A)}
$  \begin{bmatrix} 3 \\ 8 \end{bmatrix} $
\subsection*{4. B)}
$  \begin{bmatrix} 3 \\ 8 \end{bmatrix} $
\subsection*{5.}
$  \begin{bmatrix} -4\\1  \end{bmatrix} $
\subsection*{14. A)}
$ x_1  \begin{bmatrix} 2\\0\\1  \end{bmatrix} + x_2 \begin{bmatrix} 5\\1\\2  \end{bmatrix} + x_3 \begin{bmatrix} -1\\-1\\0  \end{bmatrix} = \begin{bmatrix} 4\\-1\\4  \end{bmatrix}$
This, this is a matrix equation. 
\\
\\
Converting this to REF gives 
\\
\\
$\begin{bmatrix} 2& 0& 4& 9 \\ 0& 1& -1& -1 \\ 0 & 0 & 0 & 3  \end{bmatrix}$
\\
\\
Once this is converted to a linear system we can see that the final row results in $0 = 3$, which is not possible so the system is inconsistent, meaning u does not span the columns of A. 
\subsection*{17.}
Since there is no pivot position in the 4th row of A, then by theorem 4, the equation $Ax = b$ does not have a solution for each b in the domain of $R^4$.

This goes on to show that statement D has to be false
\subsection*{19.}
Since statement D is false, then all 4 statements have to be false. Therefore, since not all vectors in $R^4$ can be written as a linear combination of the columns of A. 
\subsection*{21.}
The matrix $  \begin{bmatrix} v_1&v_2&v_3  \end{bmatrix} $ does not have a pivot in every row, thus by Theorem 4, $  \begin{bmatrix} v_1&v_2&v_3  \end{bmatrix} $ does not span $R^4$.
\subsection*{22.}
Since the system is consistent  $  \begin{bmatrix} v_1&v_2&v_3  \end{bmatrix} $ does span $R^3$.
\subsection*{24. A)}
 $Ax = b$
Since this is a vector equation and  $Ax = b$, then the matrix equation and the vector equation have the same x values. Therefore, they have the same solution set. 
\subsection*{24. B)}
Since the system is consistent, then the x values exists. Therefore, any b can be expressed by $b = a_1x_1 + a_2x_2.....$. This means that b is spanned by the columns of A. 
\subsection*{24. C)}
False
\subsection*{24. D)}
False
\subsection*{24. E)}
True
\subsection*{24. F)}
False
\subsection*{31. }
Since a does not have a pivot position in every row, the equation  $Ax = b$ cannot be consistent for all b in $R^M$
\subsection*{35.}
Suppose that y and z satisfy  $Ay = z$. This means that $4z = 4Ay$. Since that fits the definition of a scalar in an $mn$ matrix, then if we replace the constant scalar we get $4(Ay) = A(4y)$. This shows that 4y is a solution, thus making it consistent. 



\end{document}