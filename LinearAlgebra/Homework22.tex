\documentclass[12]{scrartcl}
\usepackage{amssymb,amsmath,gensymb,dsfont,calc,multicol,fullpage}
\makeatletter
\newcommand\Aboxed[1]{
   \@Aboxed#1\ENDDNE}
\def\@Aboxed#1&#2\ENDDNE{%
   &
   \settowidth\@tempdima{$\displaystyle#1{}$}
   \setlength\@tempdima{\@tempdima+\fboxsep+\fboxrule}
   \kern-\@tempdima
   \boxed{#1#2}
}
\makeatother

\begin{document}

\title{Homework 22, Section 4.4: 17, 18, 23, 27}
\author{Alex Gordon}
\date{\today}
\maketitle
\section*{Homework}
\subsection*{17.}
$0V_1 - XV_2 - XV_3$
\subsection*{18.}
$b_1 = 1 * b_1 + 0 * b_k + .... + 0 * b_n $ so the B coordinate vector of $b_1$ is $ \begin{bmatrix}  1 \\ 0 \\ ... \\ 0  \end{bmatrix} = e_1$
\subsection*{23.}
Suppose that $[u]_b = [w]_b = \begin{bmatrix}  c_1 \\ ... \\ c_n  \end{bmatrix}$. By the definition of coordinate vectors $u - w = c_1b_1 + .... + c_n b_n$. Since u and w are arbitrary elements of V the coordinate mapping is one to one. 
\subsection*{27.}
The coordinate mapping produces the coordinate vectors $(1, 0, 0, 1), (3, 1, -2, 0,), (0, -1, 3, -1) $ respectively. We now test for linear independence. 

The RREF is $\begin{bmatrix} 1 & 0 & 0 \\ 0 & 1 & 0 \\ 0 & 0 & 1 \\ 0 & 0 & 0 \end{bmatrix}$ 

Since the matrix has a pivot in each columns, it's columns (and thus the polynomials) are linearly independent. 
 

\end{document}