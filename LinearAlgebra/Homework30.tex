\documentclass[12]{scrartcl}
\usepackage{amssymb,amsmath,gensymb,dsfont,calc,multicol,fullpage}
\makeatletter
\newcommand\Aboxed[1]{
   \@Aboxed#1\ENDDNE}
\def\@Aboxed#1&#2\ENDDNE{%
   &
   \settowidth\@tempdima{$\displaystyle#1{}$}
   \setlength\@tempdima{\@tempdima+\fboxsep+\fboxrule}
   \kern-\@tempdima
   \boxed{#1#2}
}
\makeatother

\begin{document}

\title{Homework 30, Section 5.4: 7, 12, 14, 24, 25}
\author{Alex Gordon}
\date{\today}
\maketitle
\section*{Homework}
\subsection*{7.}
Since $T(b_1) = t(1) = 3 + 5t, (T(b_1))_B = \begin{bmatrix}  3 \\5 \\ 0  \end{bmatrix}.$ also, since $T(b_2) = T(t) = -2t + 4t^2, (T(b_2))_B = \begin{bmatrix}  0 \\-2 \\ 4  \end{bmatrix},$ and $T(b_3) = T(t^2) = t^2, (T(b_3))_b = \begin{bmatrix}  0 \\ 0 \\ 1  \end{bmatrix}$. Thus the matrix representation of T relative to the basis b is $\begin{bmatrix}  3 & 0 & 0  \\5 &-2 & 0 \\ 0 & 4 & 1 \end{bmatrix}$
\subsection*{12.} 
The eigenvalues of the matrix are 1 and 3. \\
For $\lambda = 1, A - I = \begin{bmatrix}  -1 & 1 \\ -3 & 3 \end{bmatrix}.$ The basis vector for the eigenspace, with $x_2$ being free, is $v_1 = \begin{bmatrix}  1\\1 \end{bmatrix}$\\
For $\lambda = 3, A - I = \begin{bmatrix}  -3 & 1 \\ -3 & 1 \end{bmatrix}.$ The basis vector for the eigenspace, with $x_2$ being free, is $v_1 = \begin{bmatrix}  1\\1 \end{bmatrix}$\\
\subsection*{14.}

\subsection*{24.}
If $A = PBP^{-1}$ then rank $A =$ rank $P(BP^{-1} =$ rank $BP^{-1}$. Also, Rank $BP^{-1} = $ rank $B$, since $p^{-1}$ is invertible. Thus, rank$A = $rank$B$.  
\subsection*{25.}
IF $A = PBP^{-1}$ then by the trace property $tr(A) = tr(P^{-1}PB) = tr(IB) = tr(B).$ if B is diagonal, then the diagonal entries of B must be the eigenvalues of A, by the diagonalization theorem. So, tr$a = $ tr $b=$ the sum of the eigenvalues of A. 



\end{document}