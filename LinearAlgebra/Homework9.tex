\documentclass[12]{scrartcl}
\usepackage{amssymb,amsmath,gensymb,dsfont,calc,multicol,fullpage}
\makeatletter
\newcommand\Aboxed[1]{
   \@Aboxed#1\ENDDNE}
\def\@Aboxed#1&#2\ENDDNE{%
   &
   \settowidth\@tempdima{$\displaystyle#1{}$}
   \setlength\@tempdima{\@tempdima+\fboxsep+\fboxrule}
   \kern-\@tempdima
   \boxed{#1#2}
}
\makeatother

\begin{document}

\title{Homework 9, Section 1.9; 8, 26, 30, 35 }
\author{Alex Gordon}
\date{\today}
\maketitle
\section*{Homework}
\subsection*{8.}
The matrix for the transformation is
$\begin{bmatrix} 0&-1 \\ -1&2  \end{bmatrix}$\\
\subsection*{26.}

\subsection*{30.}
By theorem 12, the columns of the standard matrix A must span $ \mathbb{R}^3$. 
By theorem 4, the matrix must have a pivot in each row. 
There are four possibilities for the echelon form, however they result in an inconsistent system.  Below is one. 
$\begin{bmatrix} 1& * & * & * \\ 0&1 & * & * \\ 0 & 0 & 0 & 1 \end{bmatrix}$\\ where the asterisk can be any value. 
\subsection*{35. A)}
If $ T : \mathbb{R}^n \rightarrow  \mathbb{R}^m maps  \mathbb{R}^n onto  \mathbb{R}^M, $ then its standard matrix A has a pivot in each row, by theorem 12 and 4. \\
This means that A must have at least as many columns as rows, so $ \mathbb{R}^m \leq  \mathbb{R}^n$. When T is one to one, A must have a pivot in each column by theorem 12, so $ \mathbb{R}^m \geq  \mathbb{R}^n$. 


\end{document}