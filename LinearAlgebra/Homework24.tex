\documentclass[12]{scrartcl}
\usepackage{amssymb,amsmath,gensymb,dsfont,calc,multicol,fullpage}
\makeatletter
\newcommand\Aboxed[1]{
   \@Aboxed#1\ENDDNE}
\def\@Aboxed#1&#2\ENDDNE{%
   &
   \settowidth\@tempdima{$\displaystyle#1{}$}
   \setlength\@tempdima{\@tempdima+\fboxsep+\fboxrule}
   \kern-\@tempdima
   \boxed{#1#2}
}
\makeatother

\begin{document}

\title{Homework 24, Section 4.6: 2, 6, 7, 8, 9, 10, 11, 12, 13, 14, 15, 16, 21, 24}
\author{Alex Gordon}
\date{\today}
\maketitle
\section*{Homework}
\subsection*{2.}
Rank $A = 2$ \\
$dim Nul = 2$ \\

basis for Col A is $\begin{bmatrix}  1 \\ 2 \\ 3 \\ 3  \end{bmatrix}, \begin{bmatrix}  4 \\ 6 \\ 3 \\ 0  \end{bmatrix}, \begin{bmatrix}  2 \\ -3 \\ -3 \\ 0  \end{bmatrix}$ and the Basis for Row A is (1,3,4,-1,2), (0,0,1,-1,1), (0,0,0,0,-5)


\subsection*{6.}
3, 2, 2. 
\subsection*{7.}
Yes; no. Since Col A is a four-dimensional subspace of $R^4$, it coincides with $R^4$. It can then be concluded that the null space cannot be $R^3$ because it is a different dimension, and because the vectors in Nul A have 7 entries. Nul A is a 3 dimensional subspace of $R^7$ by the rank theorem. 
\subsection*{8.}
4. It is impossible for Column A to be in $R^4$ because the vectors in Col A have 6 entries and Col A is a $R^4$ subspace of $R^6$. 
\subsection*{9.}
3, no. Since the columns of a 4 x 6 matrix are in $R^4$, rather than $R^3$,  Cal A is a 3 dimensional subspace of $R^4$. 
\subsection*{10.}
2
\subsection*{11.}
2
\subsection*{12.}
2
\subsection*{13.}
5, 5 in both cases. Since the number of pivots cannot exceed the number of columns or the number of rows, it is the same. 
\subsection*{14.}
4, 4. If A is a 5 x 4 matrix, its rows are in $R^4$ and there can be at most four linearly independent vectors in such a set. If A was a 4 x 5 matrix, then it cannot have more than four linearly independent rows because it obviously only has 4 rows. 
\subsection*{15.}
4. 
\subsection*{16.}
0
\subsection*{21.}
No. 
\subsection*{24.}
The coefficient matrix A in this case is a 7 by 6 matrix. Consider the case B in $R^7$, and the equation $Ax = b$. The equation must have a unique solution. However, since there are no free variables, the rank of A must equal the number of columns. Since the rank of A cannot exceed 6, and Col A must be a subspace of $R^7$, there must exist vectors in $R^7$ that are not in Col A. This means that the equation $Ax = b$


\end{document}