\documentclass[12]{scrartcl}
\usepackage{amssymb,amsmath,gensymb,dsfont,calc,multicol,fullpage}
\makeatletter
\newcommand\Aboxed[1]{
   \@Aboxed#1\ENDDNE}
\def\@Aboxed#1&#2\ENDDNE{%
   &
   \settowidth\@tempdima{$\displaystyle#1{}$}
   \setlength\@tempdima{\@tempdima+\fboxsep+\fboxrule}
   \kern-\@tempdima
   \boxed{#1#2}
}
\makeatother

\begin{document}

\title{Homework 1, Section 5.2: 21, 22, 23}
\author{Alex Gordon}
\date{\today}
\maketitle
\section*{Homework}
\subsection*{21. A)}
False. If a matrix is not invertible then its determinant is zero. By the invertible Matrix Theorem, then 
\subsection*{21. B)}
False. This contradicts the properties of a determinant (see theorem 3).
\subsection*{21. C)}
True. This contradicts the properties of a determinant (see theorem 3).
\subsection*{21. D)}
False. Example 4 proves this wrong. 

\subsection*{22. A)}
False. When A is a 3 x 3 matrix, det A turns out to be the volume of the parallelepiped determined by the columns of $a_1, a_2, a_3$
\subsection*{22. B)}
False. This contradicts the properties of a determinant (see theorem 3).
\subsection*{22. C)}
True. In general, the algebraic multiplicity of an eigenvalue is its multiplicity as a root of the characteristic equation. 
\subsection*{22. D)}
False. The slightly incorrect warning on page 277 shows this. 

\subsection*{23.}
If $A = QR$, with Q invertible, and if $A_1 = RQ$ then $A_1 = Q^{-1}QRQ = Q^{-1} AQ$ which shows that $A_1$ is similar to A



\end{document}