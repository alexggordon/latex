\documentclass[12]{scrartcl}
\usepackage{amssymb,amsmath,gensymb,dsfont,calc,multicol,fullpage}
\makeatletter
\newcommand\Aboxed[1]{
   \@Aboxed#1\ENDDNE}
\def\@Aboxed#1&#2\ENDDNE{%
   &
   \settowidth\@tempdima{$\displaystyle#1{}$}
   \setlength\@tempdima{\@tempdima+\fboxsep+\fboxrule}
   \kern-\@tempdima
   \boxed{#1#2}
}
\makeatother

\begin{document}

\title{Homework 20, Section 4.4: 3, 6, 14
}
\author{Alex Gordon}
\date{\today}
\maketitle
\section*{Homework}
\subsection*{3. A)}
Since $(1,3) \in R_1$ $R_1$ is not antisymmetric
\subsection*{3. B)}
Since $(1,2) \in R_2$ and  $(2,1) \in R_2$ $R_2$ is not antisymmetric
\subsection*{3. C)}
Proof. \\
Let $(x,y) \in R_3$ be given. This means that $xy +  y = y(a+1)$ is odd. If y is odd then $a + 1$ is odd. Since, they are both even though we know this is supposed to be even. This means  $(x,y) \notin R_3$ meaning $R_3$ is antisymmetric
\subsection*{6. A)}
$R_1$ is irreflexive, antisymmetric and not transitive. 
\subsection*{6. B)}
$R_2$ is irreflexive, antisymmetric and transitive. 
\subsection*{14. A)}
\
\\
\\
\\
\\
\\
\\
\\

\subsection*{14. B)}
\
\\
\\
\\
\\
\\
\\
\\
\subsection*{14. C)}
\
\\
\\
\\
\\
\\
\\
\\


\end{document}