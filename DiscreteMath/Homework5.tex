\documentclass[12]{scrartcl}
\usepackage{amssymb,amsmath,gensymb,dsfont,calc,multicol,fullpage}
\makeatletter
\newcommand\Aboxed[1]{
   \@Aboxed#1\ENDDNE}
\def\@Aboxed#1&#2\ENDDNE{%
   &
   \settowidth\@tempdima{$\displaystyle#1{}$}
   \setlength\@tempdima{\@tempdima+\fboxsep+\fboxrule}
   \kern-\@tempdima
   \boxed{#1#2}
}
\makeatother

\begin{document}

\title{Homework 5, Section 1.6: 3, 4, 5, 7}
\author{Alex Gordon}
\date{\today}
\maketitle
\section*{Homework}
\subsection*{3. A)}
The line of reasoning is invalid because of the converse fallacy
\subsection*{3. B)}
The line of reasoning is valid because of Modus Tollens
\subsection*{3. C)}
The line of reasoning is valid because of Modus Ponens
\subsection*{4. A)}
If you read the book, then you will pass the course.
\subsection*{4. B)}
If you will pass the course, then you must read the book. 
\subsection*{4. C)}
If you will pass the course, then you must read the book. 
\subsection*{5. A)}
If you want a refund, then you need a receipt. 
\subsection*{5. B)}
If you can pronounce "Euler", then you can grasp someone's mathematics background. 
\subsection*{5. C)}
If you are 16 years old, then you are legally driving in Pennsylvania. 
\subsection*{5. D)}
If you study Descartes, then you understand the history of calculus
\subsection*{5. E)}
If you want to stay dry outside, then you need to carry an umbrella
\subsection*{5. F)}
If you like horror stories, then Stephen King is fun to read. 

\subsection*{7. A)}
\begin{tabular}{ l | r | c || r || r ||}
p & q & r & $ (p \rightarrow (q \wedge r)) \vee (( p \wedge q) \rightarrow r)$ \\
  \hline                        
  T & T & T & T\\
  T & T & F & F\\
  T & F & T & T\\
  T & F & F & T\\
  F & T & T & T\\
  F & T & F & T\\
  F & F & T & T\\
  F & F & F & T\\
  \hline  
\end{tabular} \\
This expression is neither a tautology or a contradiction. 
\subsection*{7. B)}
\begin{tabular}{ l | c || r || r ||}
p & q & r & $ ((p \rightarrow q) \wedge (q \rightarrow  \neg p )) \rightarrow \neg p$ \\
  \hline                        
  T & T & T\\
  T & F & T\\
  F & T & T\\
  F & F & T\\
  \hline  
\end{tabular} \\
This expression is a tautology. 

\subsection*{7. C)}
\begin{tabular}{ l | r | c || r || r ||}
p & q & r & $ ((p \rightarrow q) \wedge (\neg p \rightarrow r)) \rightarrow (q \vee r)$ \\
  \hline                        
  T & T & T & T\\
  T & T & F & T\\
  T & F & T & T\\
  T & F & F & T\\
  F & T & T & T\\
  F & T & F & T\\
  F & F & T & T\\
  F & F & F & T\\
  \hline  
\end{tabular} \\
This expression is a tautology. 

\end{document}