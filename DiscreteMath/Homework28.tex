\documentclass[12]{scrartcl}
\usepackage{amssymb,amsmath,gensymb,dsfont,calc,multicol,fullpage}
\usepackage{float}
\restylefloat{table}
\makeatletter
\newcommand\Aboxed[1]{
   \@Aboxed#1\ENDDNE}
\def\@Aboxed#1&#2\ENDDNE{%
   &
   \settowidth\@tempdima{$\displaystyle#1{}$}
   \setlength\@tempdima{\@tempdima+\fboxsep+\fboxrule}
   \kern-\@tempdima
   \boxed{#1#2}
}
\makeatother

\begin{document}

\title{Homework 28, Section 5.3: 1, 6, 9, 12, 13, 15, 19, 28
}
\author{Alex Gordon}
\date{\today}
\maketitle
\section*{Homework}
\subsection*{1. A)}
ab, ac, ad, ae, ba, bc, bd, be, ca, cb, cd, ce, da, db, dc, de, ea, eb, ec, ed
\subsection*{1. B)}
\begin{table}[H]
    \begin{tabular}{|l|l|l|l|l|l|l|l|l|l|}
    \hline
    ab & ac & ad & ae & bc & bd & be & cd & ce & de \\
    ba & ca & da & ea & cb & db & eb & dc & ec & ed \\ \hline
    \end{tabular}
\end{table}
\subsection*{1. C)}
2, 10
\subsection*{1. D)}
10 permutations
\subsection*{6.}
You can solve the solution through placing the couples last names in a circle (there are 24 ways to do this) and then decide whether the man or the woman will stand on the right, of which there are 32 ways, meaning there are 768 ways to arrange the couples. 
\subsection*{9.}
$C(5,2)$
\subsection*{12.}
$P(6, 3) = 120$
\subsection*{13.}
$C(17, 3) = 680$
\subsection*{15. A)}
$C(21, 4) = 5,985$
\subsection*{15. B)}
$C(12,2)\cdot C(9, 2)  = 2,376$
\subsection*{15. C)}
2475
\subsection*{19. A)}
$2^8 = 256$
\subsection*{19. B)}
There are 2 possibilities
\subsection*{19. C)}
$C(8,3) = 56$
\subsection*{28. A)}
$3^6 = 729$
\subsection*{28. B)}
90 possibilities. 
\end{document}