\documentclass[12]{scrartcl}
\usepackage{amssymb,amsmath,gensymb,dsfont,calc,multicol,fullpage}
\makeatletter
\newcommand\Aboxed[1]{
   \@Aboxed#1\ENDDNE}
\def\@Aboxed#1&#2\ENDDNE{%
   &
   \settowidth\@tempdima{$\displaystyle#1{}$}
   \setlength\@tempdima{\@tempdima+\fboxsep+\fboxrule}
   \kern-\@tempdima
   \boxed{#1#2}
}
\makeatother

\begin{document}

\title{Homework 38, Section 7.7: 5, 6(b), 8(b), 9(b), 10}
\author{Alex Gordon}
\date{\today}
\maketitle
\section*{Homework}
\subsection*{5.}
The degree of the node must be greater than 2, because 1 would clearly not work. Because of this, this means that in order to use every vertex, we must have the same number of nodes in each part, meaning m and n need to be equal. Thus, as long as $m \geq 2$ then we can navigate any $k_{m,m}$
\subsection*{6. B)}
$a, d, c, e, b, a$ has the smallest weight at 24. 
\subsection*{8. B)}
The greedy algorithm produces $a, e, b, d, c, a, $ with a weight of 29
\subsection*{9. B)}
The edge-greedy algorithm adds edges in the order $[a, e], [a, c ], [b ,d ], [b ,c ] $with a cycle $a, d, c, b, e, a$ with the weight of 24
\subsection*{10.}
The vertex $a$ leads to the cycle $a, d, b, c, a$ with the weight of 1000 while the cycle weights only 100. 
 


\end{document}