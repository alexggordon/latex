\documentclass[12]{scrartcl}
\usepackage{amssymb,amsmath,gensymb,dsfont,calc,multicol,fullpage}
\makeatletter
\newcommand\Aboxed[1]{
   \@Aboxed#1\ENDDNE}
\def\@Aboxed#1&#2\ENDDNE{%
   &
   \settowidth\@tempdima{$\displaystyle#1{}$}
   \setlength\@tempdima{\@tempdima+\fboxsep+\fboxrule}
   \kern-\@tempdima
   \boxed{#1#2}
}
\makeatother

\begin{document}

\title{Homework 22, Section 3.4: 1(b,d), 4(b,d), 6(d), 10 }
\author{Alex Gordon}
\date{\today}
\maketitle
\section*{Homework }
\subsection*{1. B)}
$(ab')'= a' + (ab)$
\subsection*{1. D)}
$((a'b) + a )' = a'b'$
\subsection*{4. B)}
Prove $ab + bc = (a+c)b$\
\begin{align*} 
&=ab + bc 
\\ &=  b(a + c) \ \ distributive
\end{align*}
By the commutative property, several steps were not explained per Professor Senning saying that we really didn't have to show them. 
\subsection*{4. D)}
Prove $ab + (a' + c)' = a(b+c')$\
\begin{align*} 
&=ab + (a')')c' \ \ (DeMorgans Law)
&= ab + ac'   \ \ (Double Negative)
\\ &= a(b + c')  \ \ (Distributive)
\end{align*}
\subsection*{6. D)}
\ \\
\ \\ 
\ \\
\ \\ 
\ \\
\
\subsection*{10. A}
\ \\
\begin{table}
    \begin{tabular}{|l|l|l|l|l|l|l|l|l|}
    \hline
    ~  & 1 & 2 & 5 & 7 & 10 & 14 & 35 & 70 \\ \hline
    1  & 1 & 1 & 1 & 1 & 1  & 1  & 1  & 1  \\ \hline
    2  & 1 & 2 & 1 & 1 & 2  & 2  & 1  & 2  \\ \hline
    5  & 1 & 1 & 5 & 1 & 5  & 1  & 5  & 5  \\ \hline
    7  & 1 & 1 & 1 & 7 & 1  & 7  & 7  & 7  \\ \hline
    10 & 1 & 2 & 5 & 1 & 10 & 2  & 5  & 10 \\ \hline
    14 & 1 & 2 & 1 & 7 & 2  & 14 & 7  & 14 \\ \hline
    35 & 1 & 1 & 5 & 7 & 5  & 7  & 35 & 35 \\ \hline
    70 & 1 & 2 & 5 & 7 & 10 & 14 & 35 & 70 \\ \hline
    \end{tabular}
\end{table}
\begin{table}
    \begin{tabular}{|l|l|l|l|l|l|l|l|l|}
    \hline
    ~  & 1  & 2  & 5  & 7  & 10 & 14 & 35 & 70 \\ \hline
    1  & 1  & 2  & 5  & 7  & 10 & 14 & 35 & 70 \\ \hline
    2  & 2  & 2  & 10 & 14 & 10 & 14 & 70 & 70 \\ \hline
    5  & 5  & 10 & 5  & 35 & 10 & 70 & 35 & 70 \\ \hline
    7  & 7  & 14 & 35 & 7  & 70 & 14 & 35 & 70 \\ \hline
    10 & 10 & 10 & 10 & 70 & 10 & 70 & 70 & 70 \\ \hline
    14 & 14 & 14 & 70 & 14 & 70 & 14 & 70 & 70 \\ \hline
    35 & 35 & 70 & 35 & 35 & 70 & 70 & 35 & 70 \\ \hline
    70 & 70 & 70 & 70 & 70 & 70 & 70 & 70 & 70 \\ \hline
    \end{tabular}
\end{table}
\subsection*{10. B}
$u = 70$
\subsection*{10. C}
$z = 1$
\subsection*{10. D}
$1' = 70, 2' = 35, 5' = 14, 7' = 10, 10' = 7, 14' = 5, 35' = 2, 70' = 1$
\end{document}