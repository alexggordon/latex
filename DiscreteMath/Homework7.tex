\documentclass[12]{scrartcl}
\usepackage{amssymb,amsmath,gensymb,dsfont,calc,multicol,fullpage}
\makeatletter
\newcommand\Aboxed[1]{
   \@Aboxed#1\ENDDNE}
\def\@Aboxed#1&#2\ENDDNE{%
   &
   \settowidth\@tempdima{$\displaystyle#1{}$}
   \setlength\@tempdima{\@tempdima+\fboxsep+\fboxrule}
   \kern-\@tempdima
   \boxed{#1#2}
}
\makeatother

\begin{document}

\title{Homework 7, 2.2: 1(b,d,f), 3(a,f), 4(a), 7(b,c,d), 8, 12, 15, 21(extra credit) }
\author{Alex Gordon}
\date{\today}
\maketitle
\section*{Homework}
\subsection*{1. B)}
$187 = 17 \times 11 + 0$
\subsection*{1. D)}
$-24 = -6 \times 4 + 0$
\subsection*{1. F)}
$(9k^2 + 5) = (3k^2 + 5)3 + 1 $
\subsection*{3. A)}
$55 = 9 \times 6 + 1$
\subsection*{3. F)}
$(3k^4 - k^2 - 10k + 3)3 + 2$
\subsection*{4. A)}
If a is 12, b is 6 and c is 3, then \ $a \mod b \ and \ b \mod c \ but \ b \neq$ a
\subsection*{7. B)}
If b/a and c/a then $\frac{a^2/}{b \times c}$. First, if a divides b and a divides c, then by the closure properties b and c must be multiples of a. Again, by the closure properties, since a(a) = $a^2$ then that result is a multiple of a. If b and c are multiples of a, then by the closure properties, b $\times$ c is a multiple of a. If b(c) is a multiple of a, then it naturally follows that a divides b(c).
\subsection*{7. C)}
If a divides b and c divides d then ac divides bd. If a divides b then b must be a multiple of a. If c divides d then d must be a multiple of c. It then naturally follows that $a \times c$ is a multiple of c and and it also naturally follows that  $b times d$ is a multiple of d. Because of that, in the equation $\frac{ac}{bd}$ a/b leaves a remainder of 0 and c/d leaves a remainder of 0 and by the closure properties that means that there is a remainder of 0 meaning bd divides ac. 	
\subsection*{7. D)}
If $c$ divides $a$ then $a$ is a multiple of $c$. Therefore, for any integer x, and the problem  $\frac{ax}{cx}$ the $x$'s simply cancel out to $\frac{a(1)}{c}$
\subsection*{8.}
1) a is rational
2) b is rational
3) xw + yz
4)xw + yz
5) w $\neq$ 0
6) y $\neq$ 0
\subsection*{12.}
If $x$ is any integer, then $x = x(1)$, and so $x = \frac{x}{1}$. If x = $\frac{x}{1}$, then $x$ and 1 are both integers and 1 $\neq$ 0. Thus, $x$ can be written as a quotient of integers with a nonzero denominator, meaning $x$ is rational. 
\subsection*{15.}
$x$ is divisible by 3, thus $x = 3a$ for some integer a. Similarly, $x = 4b$ for some integer b. It then follows that if you multiply $x = 3a$ by 4 and get $4x = 12a$ and you multiply $x = 4b$ by 3 and get $3x = 12b$ then 
\begin{align*} 
x &= 4x = 3x = 12a - 12b = 12(a-b) 
\\  c &= (a-b)
\\x &= 12c
\end{align*}
Therefore, x is divisible by 12.
\subsection*{21.}
Prove that no perfect square ends in the digit 2. Let x be equal to any integer. Let  a "perfect square" = $x^2$. Since $x^2$ equals $x(x)$ then if $x(x) = y$ then the $\sqrt{y}$ must equal an integer. Therefore because the $\sqrt{2}$ is not an integer, and $x(\sqrt{y})$ is not an integer, then no perfect square root can ever end in 2. 
\end{document}
















