\documentclass[12]{scrartcl}
\usepackage{amssymb,amsmath,gensymb,dsfont,calc,multicol,fullpage}
\makeatletter
\newcommand\Aboxed[1]{
   \@Aboxed#1\ENDDNE}
\def\@Aboxed#1&#2\ENDDNE{%
   &
   \settowidth\@tempdima{$\displaystyle#1{}$}
   \setlength\@tempdima{\@tempdima+\fboxsep+\fboxrule}
   \kern-\@tempdima
   \boxed{#1#2}
}
\makeatother

\begin{document}

\title{Homework 16, Section 3.3: 5, 20(a); 4.1: 2, 5, 9(b), 13 }
\author{Alex Gordon}
\date{\today}
\maketitle
\section*{Homework: 3.3}
\subsection*{5.}
If $A \subseteq B$ and $A \subseteq C$, then it follows that $A \subseteq (B \cap C)$\\
Proof: Let A, B and C be given and assume that $A \subseteq B $ and  $A \subset C$. Let $x \in A$ be given.\\
1) If $X \in A$ and $A \subseteq B$ then $X \in B$. Similarly, it follows that if $A \subseteq C$ then $X \in C$.\\
2) If $X \in C$ and $X \in B$ the it follows that $B \cap C$ contains the element X. 
\\ Therefore, since $X \in B \cap C$, $A \subseteq (B \cap C)$
\subsection*{20. A)}
Proof: let $X \in \mathcal{P} (A \cap B)$ This means that $X \subseteq (A \cap B)$.\\
Since $X \subseteq A$, it follows that $X \in \mathcal{P}(A)$. \\
Since $\mathcal{P} (A \cap B) \subseteq \mathcal{P}(A)$, then it must be so that $X \subseteq B$. \\
Finally, it then logically shows that  $\mathcal{P} (A \cap B) \subseteq \mathcal{P}(B)$, which concludes our assertion that $\mathcal{P} (A \cap B) \subseteq \mathcal{P}(A \cap B)$
\section*{Homework: 4.1}
\subsection*{2. A)}
$g(z) = 2z + \frac{1}{z + 1}$\\
The domain is $\{z \in \mathds{R} \ | z \neq -1 \}$\\
The codomain is $\mathds{R}$
\subsection*{2. B)}
Domain: $\{\textit{t}\in\mathds{R}$ with $\textit{t}\geq 0 \}$\\
Codomain $\{\textit{y}\in\mathds{R}$ with $\textit{y}\geq 1\}$
\subsection*{2. C)}
Domain: $\{\textit{x}\in\mathds{R}$ with $\textit{x}\geq -1/2\}$\\
 Codomain $\{\textit{y}\in\mathds{R}$ with $\textit{y}\geq 0\}$
\subsection*{2. D)}
Domain: $\{\textit{t}\in\mathds{R}$ with $\textit{t}\geq -1\}$\\
 Codomain $\{\textit{y}\in\mathds{R}$ with $\textit{y}\geq 0\}$
\subsection*{2. E)}
Domain: $\mathds{R}$ \\
Codomain $\{\textit{y}\in\mathds{R}$ with 1 $\geq\textit{y}\textgreater 0 \}$

\subsection*{5.}
\
\\ \
\\ \
\\ \
\\ \
\\ \
\\ \
\\ \
\\ \
\\ \
\\ \
\\ \
\\ \
\\ \
\\ \
\\ \
\subsection*{9. B)}
\
\\ \
\\ \
\\ \
\\ \
\\ \
\\ \
\\ \
\\ \
\\ \
\\ \
\\ \
\\ \
\\ \
\\ \
\\ \
\subsection*{13.}
a is the only arrow diagram that represents a function when the domain and codomain are the set $\{1, 2, 3, 4, 5, 6\}$
\end{document}