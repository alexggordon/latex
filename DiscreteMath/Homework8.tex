\documentclass[12]{scrartcl}
\usepackage{amssymb,amsmath,gensymb,dsfont,calc,multicol,fullpage}
\makeatletter
\newcommand\Aboxed[1]{
   \@Aboxed#1\ENDDNE}
\def\@Aboxed#1&#2\ENDDNE{%
   &
   \settowidth\@tempdima{$\displaystyle#1{}$}
   \setlength\@tempdima{\@tempdima+\fboxsep+\fboxrule}
   \kern-\@tempdima
   \boxed{#1#2}
}
\makeatother

\begin{document}

\title{Homework 1, Section 2.3: 3, 5, 8(f), 10 }
\author{Alex Gordon}
\date{\today}
\maketitle
\section*{Homework}
\subsection*{3. A)}
First we must start with an outline of the proof we are about to perform. The table below includes the verification of the n = 1 through 4. It also shows the last row checked (n = m - 1) and the next row to be checked (n = m). \\ \\
\begin{tabular}{ | c | c | r | r |  }
  \hline
 $n$ & $a_n$ (recursive formula) & closed formula & equal?  \\
  \hline
  1 & 1 & $4 \times 1 - 3 = 1 $ & yes \\     \hline
  2 & $ 1 + 4 = 5$ & $4 \times 2 - 3 = 5$ & yes \\     \hline
  3 & $ 5 + 4 = 9$ & $4 \times 3 - 3 = 9$ & yes \\     \hline
  4 & $ 9 + 4 = 13$ & $4 \times 4 - 3 = 13$ & yes \\      \hline
  ... & ... & ... & ... \\   \hline
 $m - 1$ & $a_{m-2} + 4 = 4m - 7$ & $4(m - 1) - 3 = 4m - 7$ & yes \\   \hline \hline
  $m$ & $a_{m-1} + 4$& $4m - 3$ & ??? \\                  
  \hline  
\end{tabular}
\\
\\All let is left is to simplify the recursive formula for the closed formula. We can do this by some simple substitution. 
\begin{align*} 
a_m &= a_{m-1} + 4
\\ a_{m-1} &= (4m - 7)
\\ a_m &= (4m - 7) + 4
\\ &= 4m - 3
\end{align*}
This now shows that the recursive formula is equal to the closed formula. 
\subsection*{3. B)}
First we must start with an outline of the proof we are about to perform. The table below includes the verification of the n = 1 through 4. It also shows the last row checked (n = m - 1) and the next row to be checked (n = m). \\ \\
\begin{tabular}{ | c | c | r | r |  }
  \hline
 $n$ & $a_n$ (recursive formula) & closed formula & equal?  \\
  \hline
  1 & 5 & $\frac{1(1+9)}{2} = 5 $ & yes \\     \hline
  2 & $5 + 2 + 4 = 11$ & $\frac{2(2+9)}{2} = 11$ & yes \\     \hline
  3 & $11 + 3 + 4 = 18$ & $\frac{3(3+9)}{2} = 18$ & yes \\     \hline
  4 & $18 + 4 + 4 = 26$ & $\frac{4(4+9)}{2} = 26$ & yes \\      \hline
  ... & ... & ... & ... \\   \hline
 $m - 1$ & $ a_{m-2} + a_{m-1} + 4 = \frac{(m-1)((m)+9)}{2}$ & $ \frac{(m-1)(m-1) + 9}{2}  = \frac{m(m+9)}{2} $ & yes \\   \hline \hline
  $m$ & $a_{m-1} + a_{m} + 4$& $\frac{4(m)}{2}$ & ??? \\                        
  \hline  
\end{tabular}
\\
\\All let is left is to simplify the recursive formula for the closed formula. We can do this by some simple substitution. 
\begin{align*} 
a_m &= \frac{(m-1)((m)+9)}{2}
\\ a_{m-1} &= \frac{m(m+9)}{2}
\\ &= \frac{m^2 + 9m}{2}
\end{align*}
This now shows that the recursive formula is equal to the closed formula. 
\subsection*{3. C)}
First we must start with an outline of the proof we are about to perform. The table below includes the verification of the n = 1 through 4. It also shows the last row checked (n = m - 1) and the next row to be checked (n = m). \\ \\
\begin{tabular}{ | c | c | r | r |  }
  \hline
 $n$ & $a_n$ (recursive formula) & closed formula & equal?  \\
  \hline
  1 & 1 & $\frac{1(2)(3)}{6} = 1 $ & yes \\     \hline
  2 & $1 + 2^2 = 5$ & $\frac{2(3)(5)}{6} = 5$ & yes \\     \hline
  3 & $5 + 3^2 = 14$ & $\frac{3(4)(7)}{2} = 14$ & yes \\     \hline
  4 & $14 + 4^2 = 30$ & $\frac{4(5)(9)}{2} = 30$ & yes \\      \hline
  ... & ... & ... & ... \\   \hline
 $m - 1$ & $ a_{m-2} + (m-1)^2 = \frac{(m-1)(m)(2m-1)}{6}$ & $\frac{(m-1)(m)(2m-1)}{6}$ & yes \\   \hline \hline
  $m$ & $a_{m-1} + m^2$& $\frac{(m)(m+1)(2m+1)}{6}$ & ??? \\                           
  \hline  
\end{tabular}
\\
\\All let is left is to simplify the recursive formula for the closed formula. We can do this by some simple substitution. 
\begin{align*} 
a_m &= a_{m-1}  + m^2
\\  &= \frac{(m-1)(m)(2m-1)}{6} + m^2
\\  &= \frac{(m-1)(m)(2m-1) +6m^2}{6}
\\ &= \frac{m[(m-1)(2m-1) +6m]}{6}
\\ &= \frac{m(m+1)(2m+1)}{6}
\end{align*}
This now shows that the recursive formula is equal to the closed formula. 
\subsection*{3. D)}
First we must start with an outline of the proof we are about to perform. The table below includes the verification of the n = 1 through 4. It also shows the last row checked (n = m - 1) and the next row to be checked (n = m). \\ \\
\begin{tabular}{ | c | c | r | r |  }
  \hline
 $n$ & $a_n$ (recursive formula) & closed formula & equal?  \\
  \hline
  1 & 1 & $2^1 = 1 $ & yes \\     \hline
  2 & $1 + 2^2 = 5$ & $2^2 - 1= 3$ & yes \\     \hline
  3 & $5 + 3^2 = 14$ & $2^3 - 1= 7$ & yes \\     \hline
  4 & $14 + 4^2 = 30$ & $2^4 - 1= 15$ & yes \\      \hline
  ... & ... & ... & ... \\   \hline
 $m - 1$ & $ 2 (a_{m-2}) + 1 = 2^{m-1} - 1$ & $2^{m-1} - 1$ & yes \\   \hline \hline
  $m$ & $2 (a_{m-1}) + 1$& $2^m-1$ & ??? \\                           
  \hline  
\end{tabular}
\subsection*{3. E)}

\subsection*{3. F)}

\subsection*{5. A)}

\subsection*{10. A)}

$\sum\limits_{i=1}^n = \frac{i(i+1)}{2}= \frac{n(n+1)(n+2)}{6}$

\end{document}