\documentclass[12]{scrartcl}
\usepackage{amssymb,amsmath,gensymb,dsfont,calc,multicol,fullpage}
\usepackage{float}
\restylefloat{table}
\makeatletter
\newcommand\Aboxed[1]{
   \@Aboxed#1\ENDDNE}
\def\@Aboxed#1&#2\ENDDNE{%
   &
   \settowidth\@tempdima{$\displaystyle#1{}$}
   \setlength\@tempdima{\@tempdima+\fboxsep+\fboxrule}
   \kern-\@tempdima
   \boxed{#1#2}
}
\makeatother

\begin{document}

\title{Homework 29, Section 5.4 2, 7, 10, 14, 18 (a formula answer is sufficient), 20, and the problem below.}
\author{Alex Gordon}
\date{\today}
\maketitle
\section*{Homework}
\subsection*{2. A)}
$C(8,6) = 28 possibilities$
\subsection*{2. B)}
$Same question, so 28$
\subsection*{2. C)}
$2^8 - C(8,1) = 247$
\subsection*{7.}
$C(4,4) \cdot C(7,3) \cdot C(10,3) = 4,200$
\subsection*{10.}
$C(11,2) \cdot C(9,2) \cdot 5 \cdot C(7,2) = 9,989,600$
\subsection*{14.}
\begin{table}[H]
    \begin{tabular}{|l|l|l|}
    \hline
    Fruit  & Equation        & Binary Sequence \\ \hline
    5a, 5b & 5 + 5 + 0 = 10  & 000001000001    \\ \hline
    5a, 5p & 5 + 5 + 0 = 10  & 000001100000    \\ \hline
    1a, 9p & 1 + 0 + 9 = 10  & 011000000000    \\ \hline
    10 b   & 0 + 10 + 0 = 10 & 100000000001    \\ \hline
    \end{tabular}
\end{table}
\subsection*{18.}
$C(49,20) =$ Some really big number I'm too lazy to copy. 
\subsection*{The "Problem below" A)}
$C(13,6) =1716$
\subsection*{The "Problem below" B)}
$C(19,12) =50,388$
\subsection*{The "Problem below" C)}
$C(31,24) = Some big number$
\subsection*{The "Problem below" D)}
$C(11,4) = 330$
\subsection*{The "Problem below" E)}
$C(15,9) = 5005$
\end{document}