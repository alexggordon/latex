\documentclass[12]{scrartcl}
\usepackage{amssymb,amsmath,gensymb,dsfont,calc,multicol,fullpage}
\makeatletter
\newcommand\Aboxed[1]{
   \@Aboxed#1\ENDDNE}
\def\@Aboxed#1&#2\ENDDNE{%
   &
   \settowidth\@tempdima{$\displaystyle#1{}$}
   \setlength\@tempdima{\@tempdima+\fboxsep+\fboxrule}
   \kern-\@tempdima
   \boxed{#1#2}
}
\makeatother

\begin{document}

\title{Homework 10, Section 2.5: 6, 12, 14, 15, 23}
\author{Alex Gordon}
\date{\today}
\maketitle
\section*{Homework}
\subsection*{6.}
Contradicting statement: There can be an integer that is both even and odd. \\ \\
First off, let me define and odd and even integer. An even integer is one that can be defined in the form $2j$ where $j$ is any integer. Let an odd integer be defined in the form $2n + 1$. Given these definitions, if an integer is both even and odd, then it must match the form\\
\begin{center}$2n = 2n+1$\end{center} 
Since no number can ever equal itself plus one, that statement is not true, meaning the contradicting statement, "there is no integer that can be both even and odd" is true. 
\subsection*{12.}
Let $a$ be a rational number and let $b$ be an irrational number. Assume that $a + b = c$ contains all rational numbers. 
\\
If this is the case there must exist integers $a$, $b$ and $c$ such that $a = \frac{c}{b}$, and $a$ and $c$ can be chosen to be relatively prime (no factors in common). This must also be true then that $b^2 = \frac{c}{b}^2$
\subsection*{14. A)}
\begin{center}$n = \frac{5}{3}$\end{center}
\subsection*{14. B)}
\begin{center}$n = 1,000,000,000$\end{center}
\subsection*{14. C)}
\begin{center}$n = 8$\end{center}
\subsection*{15.}
Contradicting statement: The $\sqrt{5}$ is rational. 
Suppose that $\sqrt{5}$ is rational. That means $\sqrt{5} = \frac{a}{b}$ where $a$ and $b$ are some integers that are relatively prime. This means that if we were to square both sides then $(\sqrt{5})^2 = (\frac{a}{b})^2$. It then follows that
\begin{align*} 
(\sqrt{5})^2 &= \frac{a^2}{b^2}
\\ 5 &= \frac{a^2}{b^2}
\\ 5b^2 &= a^2\\
\end{align*}

If $a$ and $b$ are in lowest terms (as supposed), their squares would each have an even number of prime factors. $5b^2$ has one more prime factor than $b^2$, meaning it would have an odd number of prime factors. Every composite has a unique prime factorization and can't have both an even and odd number of prime factors, this shows that $\sqrt{5}$ is not rational. 

\subsection*{23.}
Contradicting statement: If the average net weight of seven boxes of cereal is 17 ounces, then at least one of the boxes does not have a net weight of at least 17 ounces. \\
\\
Let $k_1, … k_7$ represent the weight of the 7 cereal boxes. Of $k_1, ... k_7$, since none of them have a net weight of at least 17 ounces, we know that their sum is less than 119 ounces. This means that \\
\begin{center} $k_1 + k_2 + k_3... k_7 \  \textless \ 119$ \end{center}
However, based on the proposition that the average weight is 17 ounces, then 
\begin{center} $k_1 + k_2 + k_3... k_7 \  \geq \ 119$ \end{center} 
and it also means that 
\begin{center} $\frac{119} {k_1 + k_2 + k_3... k_7}  \leq 1$ \end{center} 
This means at least one of the boxes is 17 ounces because if they were all less that 17 ounces then
\begin{center} $\frac{119} {k_1 + k_2 + k_3... k_7}  \geq 1$ \end{center} 
which is not true. 
\end{document}