\documentclass[12]{scrartcl}
\usepackage{amssymb,amsmath,gensymb,dsfont,calc,multicol,fullpage}
\makeatletter
\newcommand\Aboxed[1]{
   \@Aboxed#1\ENDDNE}
\def\@Aboxed#1&#2\ENDDNE{%
   &
   \settowidth\@tempdima{$\displaystyle#1{}$}
   \setlength\@tempdima{\@tempdima+\fboxsep+\fboxrule}
   \kern-\@tempdima
   \boxed{#1#2}
}
\makeatother

\begin{document}

\title{Homework 18, Section 4.3: 3, 7, 8, 15}
\author{Alex Gordon}
\date{\today}
\maketitle
\section*{Homework}
\subsection*{3. A)}
It is invertible because the function is one-to-one and onto.
\subsection*{3. B)}
One to one, but not onto
\subsection*{3. C)}
One to one but not onto
\subsection*{3. D)}
It is invertible because the function is one-to-one and onto.

\subsection*{7.}
$f$($x_{1}$) = $\textit{f}$($\textit{x}_{2}$)\\
$\textit{f}$ is one-to-one\\
2$\textit{x}_{1}$ = 2$\textit{x}_{2}$\\
Therefore\\
$\textit{x}_{1}$ = $\textit{x}_{2}$

\subsection*{8.}
Proof: 
 since $\textit{f}$ is onto then there must be an $x \in \mathds{R}$ with $f(x) = y.$ If we take $z = \frac{x}{2} \in \mathds{R}$ it follows that $h(z) = f(2z) = f(x)$\\ This shows that $h(z) = y$
\subsection*{15. A)}
Proof\\
let$x_1 and x_2 \in  \mathds{R}$ be given. \\
Let it also be given that $y(x_1) = y(x_2)$. This means that given the function, $5x_1 = 7 = 5x_2 = 7$\\
This in turn means $x_1 = x_2$ meaning y is one to one. 
\subsection*{15. B)}

\subsection*{15. C)}

\end{document}